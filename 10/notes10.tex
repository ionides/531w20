%\documentclass[handout]{beamer}
\documentclass{beamer}\usepackage[]{graphicx}\usepackage[]{color}
% maxwidth is the original width if it is less than linewidth
% otherwise use linewidth (to make sure the graphics do not exceed the margin)
\makeatletter
\def\maxwidth{ %
  \ifdim\Gin@nat@width>\linewidth
    \linewidth
  \else
    \Gin@nat@width
  \fi
}
\makeatother

\definecolor{fgcolor}{rgb}{0.345, 0.345, 0.345}
\newcommand{\hlnum}[1]{\textcolor[rgb]{0.686,0.059,0.569}{#1}}%
\newcommand{\hlstr}[1]{\textcolor[rgb]{0.192,0.494,0.8}{#1}}%
\newcommand{\hlcom}[1]{\textcolor[rgb]{0.678,0.584,0.686}{\textit{#1}}}%
\newcommand{\hlopt}[1]{\textcolor[rgb]{0,0,0}{#1}}%
\newcommand{\hlstd}[1]{\textcolor[rgb]{0.345,0.345,0.345}{#1}}%
\newcommand{\hlkwa}[1]{\textcolor[rgb]{0.161,0.373,0.58}{\textbf{#1}}}%
\newcommand{\hlkwb}[1]{\textcolor[rgb]{0.69,0.353,0.396}{#1}}%
\newcommand{\hlkwc}[1]{\textcolor[rgb]{0.333,0.667,0.333}{#1}}%
\newcommand{\hlkwd}[1]{\textcolor[rgb]{0.737,0.353,0.396}{\textbf{#1}}}%
\let\hlipl\hlkwb

\usepackage{framed}
\makeatletter
\newenvironment{kframe}{%
 \def\at@end@of@kframe{}%
 \ifinner\ifhmode%
  \def\at@end@of@kframe{\end{minipage}}%
  \begin{minipage}{\columnwidth}%
 \fi\fi%
 \def\FrameCommand##1{\hskip\@totalleftmargin \hskip-\fboxsep
 \colorbox{shadecolor}{##1}\hskip-\fboxsep
     % There is no \\@totalrightmargin, so:
     \hskip-\linewidth \hskip-\@totalleftmargin \hskip\columnwidth}%
 \MakeFramed {\advance\hsize-\width
   \@totalleftmargin\z@ \linewidth\hsize
   \@setminipage}}%
 {\par\unskip\endMakeFramed%
 \at@end@of@kframe}
\makeatother

\definecolor{shadecolor}{rgb}{.97, .97, .97}
\definecolor{messagecolor}{rgb}{0, 0, 0}
\definecolor{warningcolor}{rgb}{1, 0, 1}
\definecolor{errorcolor}{rgb}{1, 0, 0}
\newenvironment{knitrout}{}{} % an empty environment to be redefined in TeX

\usepackage{alltt}


\usefonttheme[onlymath]{serif}
% use article-like math letters, from http://tex.stackexchange.com/questions/34265/how-to-get-beamer-math-to-look-like-article-math

\newcommand\code[1]{\url{#1}}
\newcommand\question{{\bf Question}}
\newcommand\mysolution{{\bf Solution}}
\newcounter{Qcounter}
\newcommand\myquestion{{\stepcounter{Qcounter} \bf Question \CHAPTER.\theQcounter}}
\newcounter{WorkedExampleCounter}
\newcommand\WorkedExample{{\stepcounter{WorkedExampleCounter} \bf Worked example \CHAPTER.\theWorkedExampleCounter}}

\newcommand\myexample{{\bf Example}}
\newcommand\mydot{{\,\cdot\,}}
\newcommand\myref[1]{\m{#1}}
%\newcommand\mynotes[2]{#1}
\newcommand\mynotes[2]{#2}
\newcommand\Rspace{\mathcal{R}}
\newcommand\Sspace{\mathcal{S}}
\usepackage{amsmath}
\renewcommand\vec[1]{\boldsymbol{\mathrm{#1}}}
\newcommand\vect[1]{\vec{#1}}
\newcommand\mat[1]{\mathbb{#1}}
%\newcommand\mat[1]{\mathcal{#1}}
\newcommand\mymatrix[3]{\left[
\begin{array}{cccc}
{#1}_{11} & {#1}_{12} & \dots & {#1}_{1{#3}} \\
{#1}_{21}& {#1}_{22} & \dots & {#1}_{2{#3}} \\ 
\vdots & \vdots & \ddots & \vdots \\
{#1}_{{#2}1} & {#1}_{{#2}2} & \dots & {#1}_{{#2}{#3}} 
\end{array}
\right]
}
\newcommand\myvector[2]{\left[
\begin{array}{c}
{#1}_{1} \\
{#1}_{2} \\
\vdots \\
{#1}_{{#2}}
\end{array}
\right]
}
\newcommand\mytwovector[2]{\left[
\begin{array}{c}
{#1} \\
{#2}
\end{array}
\right]
}
\newcommand\mytwomatrix[4]{\left[
\begin{array}{cc}
{#1} & {#2} \\
{#3} & {#4}
\end{array}
\right]
}
\newcommand\mysmallmatrix[3]{\left[
\begin{array}{ccc}
{#1}_{11} & \dots & {#1}_{1{#3}} \\
\vdots & \ddots & \vdots \\
{#1}_{{#2}1} & \dots & {#1}_{{#2}{#3}} 
\end{array}
\right]
}


\newcommand\prob{\mathbb{P}}
\newcommand\E{\mathbb{E}}
%\newcommand\lik{\mathscr{L}}
\newcommand\lik{\mathcal{L}}
\newcommand\loglik{\ell}
\newcommand\profileloglik[1]{\ell^\mathrm{profile}_#1}
\newcommand\R{\mathbb{R}}
%\newcommand\data[1]{#1^*}
\newcommand\data[1]{#1}
\newcommand\estimate[1]{\data{#1}}
\newcommand\bi{\begin{itemize}}
\newcommand\ei{\end{itemize}}
\newcommand\normal{\mathrm{normal}}
\newcommand\iid{\mathrm{iid}}
\newcommand\MVN{\mathrm{MVN}}
\newcommand\SE{\mathrm{SE}}
%\newcommand\SD{\mathrm{SD}}
\newcommand\RSS{\mathrm{RSS}}
\newcommand\SST{\mathrm{SST}}
\newcommand\pval{\mathrm{pval}}
\newcommand\var{\mathrm{Var}}
\newcommand\sd{\mathrm{SD}}
\newcommand\sdSample{\mathrm{sd}}
\newcommand\varSample{\mathrm{var}}
\newcommand\cov{\mathrm{Cov}}
\newcommand\covSample{\mathrm{cov}}
\newcommand\corSample{\mathrm{cor}}
\newcommand\cor{\mathrm{Cor}}
\newcommand\given{{\, | \,}}
\newcommand\param{\,;}
\newcommand\params{\param}
\newcommand\transpose{{\raisebox{0.5mm}{\mbox{\scriptsize \textsc{t}}}}}
%\newcommand\transpose{\scriptsize{T}}
\newcommand\mycolon{{\hspace{0.5mm}:\hspace{0.5mm}}}
%\newcommand\mycolon{\,{:}\,}
\newcommand\myemph[1]{{\textbf{#1}}}
\newcommand\mymathenv[1]{\textcolor{blue}{#1}}
\newcommand\mymath[1]{\begin{math}\textcolor{blue}{#1}\end{math}}
\newcommand\m[1]{\mymath{#1}}
\newcommand\mydisplaymath[1]{\begin{displaymath}\textcolor{blue}{#1}\end{displaymath}}
\newcommand\myeqnarray[1]{\textcolor{blue}{\begin{eqnarray*}#1 \end{eqnarray*}}}
\newcommand\myspace{\quad}
\newcommand\altdisplaymath[1]{\vspace{1mm}\textcolor{blue}{\begin{math}\displaystyle #1 \end{math}}\vspace{1mm}}
\usepackage{natbib}
\usepackage{url}
%\usepackage{ulem}
%\renewcommand\emph[1]{{\it #1}} % the ulem package redefines \emph
%\renewcommand\em{\it} % the ulem package redefines \emph




\newcommand\enumerateSpace{\hspace{2mm}}
\usepackage{amssymb}
\newenvironment {myitemize} {
                 \begin{list}{\textcolor{black}{$\bullet$} \hfill}
%                 \begin{list}{\textcolor{blue}{{\small{$\blacktriangleright$}}} \hfill}
                 {\setlength{\labelwidth}{0.3 cm}
                  %\setlength{\leftmargin}{0em}
                  \setlength{\leftmargin}{0.15cm}
                  \setlength{\itemindent}{0.15cm}
                  \setlength{\labelsep}{0cm}
                  \setlength{\parsep}{0.2 ex}
%                  \setlength{\itemsep}{0.25 cm}
                  \setlength{\itemsep}{1 mm}
%                   \setlength{\itemsep}{0.0 cm}
      \setlength{\topsep}{0.0cm}}} %space between title and 1st item
   {\end{list}}

\usepackage{graphicx} % Allows including images
%\usepackage{booktabs} % Allows the use of \toprule, \midrule and \bottomrule in tables
\mode<presentation> {

\usetheme{Madrid}

\setbeamertemplate{footline} 

\setbeamertemplate{navigation symbols}{} 

}

\setlength{\parskip}{0mm}
\setlength{\parindent}{0mm}
%\newcommand\negBeforeCode{\vspace{-2mm}}
%\newcommand\negAfterCode{\vspace{-3mm}}
\newcommand\negBeforeCode{}
\newcommand\negAfterCode{}

%\renewenvironment{knitrout}{\vspace{-3mm}}{\vspace{-5mm}}






\newcommand\CHAPTER{10}


\newcommand\eqspace{\quad\quad\quad}
\newcommand\eqskip{\vspace{2.5mm}}

\newcommand\myeq{\hspace{10mm}}

% \newcommand\answer[2]{\textcolor{blue}{#2}} % to show answers
% \newcommand\answer[2]{{\color{blue}{#2}}} % to show answers
\newcommand\answer[2]{#1} % to show blank space



\usepackage{bbm} % for blackboard bold 1
\IfFileExists{upquote.sty}{\usepackage{upquote}}{}
\begin{document}

% knitr set up




\begin{frame}[fragile]
\frametitle{Chapter \CHAPTER. Statistical methodology for nonlinear partially observed Markov process models}

\hspace{3cm} {\large \bf Objectives}

\vspace{3mm}

\begin{enumerate}

\item To introduce students to the \package{pomp} package

\item To explain how the components of a POMP model are encoded in this package


\item To give some experience in the use and manipulation of \code{pomp} objects.

\end{enumerate}

\end{frame}

\begin{frame}[fragile]

\frametitle{Time series analysis via nonlinear partially observed Markov process (NL-POMP) models}

Six problems of Bjornstad and Grenfell (Science, 2001): obstacles for ecological modeling and inference via nonlinear mechanistic models:

\begin{enumerate}

\item Combining measurement noise and process noise.

\item Including covariates in mechanistically plausible ways.

\item Continuous time models.

\item Modeling and estimating interactions in coupled systems. 

\item Dealing with unobserved variables.

\item Modeling spatial-temporal dynamics.

\end{enumerate}

\end{frame}  

\begin{frame}[fragile] 

\frametitle{Applications of NL-POMP models}

\bi

\item Modeling and inference via nonlinear mechanistic models arises throughout engineering, the sciences (social, biological and physical) and business.

\item In finance, we considered a stochastic volatility example in Chapter 1.

\item Infectious disease transmission dynamics can be highly nonlinear. Transmission arises when an infected individual contacts a susceptible individual, thus the rate of infections gets a nonlinear term:
$$\begin{array}{ll}
\mbox{Fraction of individuals infected}\\
\quad\times\quad \mbox{Fraction of individuals susceptible to infection}
\end{array}$$

\item Each new infection depletes the pool of susceptible individuals.  Without depletion of susceptibles, the fraction of individuals susceptible to infection is constant and the epidemic grows exponentially. 

\item Data on infectious diseases are generally limited to diagnosed cases. Much of the transmission dynamics cannot be directly observed. 

\ei

\end{frame}

\begin{frame}[fragile]

\frametitle{Ecology and epidemiology as historical motivation for NL-POMP methods}

\bi

\item Infectious disease epidemiology has motivated developments in statistical methodology and software for NL-POMP models.

\item Many other biological populations have similar nonlinearities: the population grows exponentially until limited by some constraint, such as a food resource or a predator. When the resource is used up, or the predator becomes abundant, the population crashes. Then a new cycle begins.

\item Ecological systems and epidemiological systems have much in common. Disease systems can be relatively simple ecosystems with just two species: host and pathogen.

\ei

\end{frame}   

\begin{frame}[fragile] 

\frametitle{An algorithmic approach to inference for POMP models}

Recall our notation for partially observed Markov process models. The latent process is $X_n=X(t_n)$ and we write $X_{0:N}=(X_0,\dots,X_N)$. The observation at time $t_n$ is modeled by a random variable $Y_n$. The one-step transition density, $f_{X_n|X_{n-1}}(x_n|x_{n-1};\theta)$, together with the measurement density, $f_{Y_n|X_n}(y_n|x_n;\theta)$ and the initial density, $f_{X_0}(x_0;\theta)$, specify the entire joint density via
$$f_{X_{0:N},Y_{1:N}}(x_{0:N},y_{1:N};\theta) = f_{X_0}(x_0;\theta)\,\prod_{n=1}^N\!f_{X_n | X_{n-1}}(x_n|x_{n-1};\theta)\,f_{Y_n|X_n}(y_n|x_n;\theta).$$
The marginal density for $Y_{1:N}$, evaluated at the data, $\data{y_{1:N}}$, is
$$ f_{Y_{1:N}}(\data{y}_{1:N}\param\theta)=\int f_{X_{0:N},Y_{1:N}}(x_{0:N},\data{y}_{1:N}\param\theta)\, dx_{0:N}.
$$


\vspace{2mm}

\end{frame}

\begin{frame}[fragile]

\frametitle{Algorithmic notation matching the mathematical notation for POMP models}

To think algorithmically, we define some function calls that provide \myemph{basic elements} specifying a POMP model.

\bi
\item \code{rprocess( )}: a draw from the one-step transition distribution, with density $f_{X_n|X_{n-1}}(x_n| x_{n-1};\theta)$.

\item \code{dprocess( )}: evaluation of the one-step transition density, $f_{X_n|X_{n-1}}(x_n| x_{n-1};\theta)$.

\item \code{rmeasure( )}: a draw from the measurement distribution with density $f_{Y_n|X_n}(y_n| x_n;\theta)$.

\item \code{dmeasure( )}: evaluation of the measurement density, $f_{Y_n|X_n}(y_n| x_n;\theta)$.

\item This follows the standard R notation, for example we expect \code{rnorm} to draw from the normal distribution, and \code{dnorm} to evaluate the normal density.

\item A general POMP model is fully specified by defining these basic elements.

\ei

\end{frame}

\begin{frame}[fragile]

\frametitle{Specifying our own POMP model}

\bi

\item The user will have to say what the basic elements are for their chosen POMP model.

\item Algorithms can then use these basic elements to carry out inference for the POMP model.

\item We will see that there are algorithms that can carry out likelihood-based inference for this general POMP model specification.

\ei

\end{frame}  

\begin{frame}[fragile] 

\frametitle{What does it mean for statistical methodology to be simulation-based?}

\bi

\item Simulating random processes can be easier than evaluating their transition probabilities.

\item Thus, we may be able to write \code{rprocess()} but not \code{dprocess()}.

\item  \myemph{Simulation-based} methods require the user to specify \code{rprocess()} but not \code{dprocess()}.
 
\item \myemph{Plug-and-play}, \myemph{likelihood-free} and \myemph{equation-free} are alternative terms for simulation-based.

\item Much development of simulation-based statistical methodology has occurred in the past decade.

\ei

\end{frame}    

\begin{frame}[fragile] 

\frametitle{The \myemph{pomp}  {\R}  package for POMP models}

\bi

\item \package{pomp} is an  {\R}  package for data analysis using partially observed Markov process (POMP) models.

\item Note the distinction: lower case \package{pomp} is a software package; 
upper case POMP is a class of models.

\item \package{pomp} builds methodology for POMP models in terms of arbitrary user-specified \code{rprocess()}, \code{dprocess()}, \code{rmeasure()}, and \code{dmeasure()} functions.
 
\item Following modern practice, most methodology in \package{pomp} is simulation-based, so does not require specification of \code{dprocess()}.

\item \package{pomp} has facilities to help construct \code{rprocess()}, \code{rmeasure()}, and \code{dmeasure()} functions for model classes of scientific interest.

\item \package{pomp} provides a forum for development, modification and sharing of models, methodology and data analysis workflows.

\item \package{pomp} is available from CRAN or github.

\ei

\end{frame}    

\begin{frame}[fragile] 

\frametitle{Example: the Ricker model}


\vspace{-1mm}

\bi

\item The Ricker model is a basic model in population biology. We start with a deterministic version and then add process noise and measurement error.


\item The \myemph{Ricker equation} is a deterministic differential equation modeling the dynamics of a simple population, including population growth and resource depletion.
\ei
\vspace{1mm}

[R1] $ \myeq P_{n+1} = r\,P_{n}\,\exp(-P_{n}).$

\vspace{1mm}
\bi
\item Here, $P_n$ is the population density at time $t_n=n$ and $r$ is a fixed value (a parameter), related to the population's intrinsic capacity to increase.

\item $P_n=\log(r)$ is an \myemph{equilibrium}, meaning that if $P_n=\log(r)$ then $P_{n+1}=P_{n+2}=\dots = P_n$. Another equilibrium is $P_n=0$. 

\item  $P$ is a \myemph{state variable}, $r$ is a \myemph{parameter}.

\item If we know $r$ and the \myemph{initial condition} $P_0$, this deterministic Ricker equation predicts the future population density at all times.

\item The initial condition, $P_0$ is a special kind of parameter, an \myemph{initial-value parameter}.
\ei

\end{frame}   

\begin{frame}[fragile] 

\frametitle{Adding stochasticity to the Ricker equation}

\bi

\item We can model process noise by making the growth rate $r$ into a random variable.

\item For example, if we assume that the intrinsic growth rate is log-normally distributed, $P$ becomes a stochastic process governed by

\ei

\vspace{2mm}

[R2] $\myeq P_{n+1} = r\,P_{n}\,\exp(-P_{n}+\varepsilon_{n}), \qquad \varepsilon_{n}\;\sim\;\mathrm{Normal}(0,\sigma^2)$,

\vspace{1mm}

\bi

\item Here, the new parameter $\sigma$ is the standard deviation of the noise process $\varepsilon$.

\ei

\end{frame}    

\begin{frame}[fragile] 

\myquestion. Does adding Gaussian noise mean we have a Gaussian latent process model? What does it mean to say that the model for $P_{0:N}$ described by equation [R2] is Gaussian?

\answer{\vspace{60mm}}{todo}

\end{frame}    

\begin{frame}[fragile] 
\frametitle{Adding measurement error to the Ricker model}

\bi
\item Let's suppose that the Ricker model is our model for the dynamics of a real population.

\item For most populations, outside of controlled experiments, we cannot know the exact population density at any time, but only estimate it through sampling.

\item Let's model measurement error by treating the measurement $\data{y_{n}}$, conditional on $P_n$, as a draw from a Poisson distribution with mean $\phi\,P_n$. This corresponds to the model

\ei

\vspace{2mm}

[R3] $\myeq Y_{n}|P_n\;\sim\;\mathrm{Poisson}(\phi\,P_{n})$.

\vspace{1mm}

\bi

\item The parameter $\phi$ is proportional to the sampling effort.

\ei

\end{frame}

\begin{frame}[fragile]

\frametitle{Writing the Ricker model as a POMP model}

\bi
\item For our standard definition of a POMP model $(X_{0:N},Y_{0:N})$, we can check that equations [R2] and [R3] together with the parameter $P_0$ define a POMP model with
\begin{eqnarray}
X_n &=& P_n \\
Y_n &=& Y_n 
\end{eqnarray}

\item Following the usual POMP paradigm, $P_n$ is a true but unknown population density at time $t_n$.

\ei

\end{frame}    

\begin{frame}[fragile] 

\frametitle{Working with the Ricker model in \package{pomp}}

\bi

\item The  R  package \package{pomp} provides facilities for modeling POMPs, a toolbox of statistical inference methods for analyzing data using POMPs, and a development platform for implmenting new POMP inference methods.

\item The basic data-structure provided by \package{pomp} is the object of class \code{pomp}, alternatively known as a \code{pomp} object.

\item A \code{pomp} object is a container that holds real or simulated data and a POMP model, possibly together with other information such as model parameters, that may be needed to do things with the model and data.

\item Let's see what can be done with a \code{pomp} object.

\item First, if we haven't already, we must install \package{pomp}. The package needs access to code compilers to operate properly so you should check the installation instructions at
\url{https://kingaa.github.io/pomp/install.html}

\ei

\end{frame}

\begin{frame}[fragile]
\bi
\item For the following, we also load some other packages.
\ei
\begin{knitrout}\small
\definecolor{shadecolor}{rgb}{0.969, 0.969, 0.969}\color{fgcolor}\begin{kframe}
\begin{alltt}
\hlkwd{set.seed}\hlstd{(}\hlnum{594709947L}\hlstd{)}
\hlkwd{require}\hlstd{(ggplot2)}
\hlkwd{require}\hlstd{(plyr)}
\hlkwd{require}\hlstd{(reshape2)}
\hlkwd{require}\hlstd{(pomp)}
\end{alltt}
\end{kframe}
\end{knitrout}

\end{frame}

\begin{frame}[fragile]
\bi
\item A pre-built \code{pomp} object encoding the Ricker model comes included with the package.
Load it by
\ei

\begin{knitrout}\small
\definecolor{shadecolor}{rgb}{0.969, 0.969, 0.969}\color{fgcolor}\begin{kframe}
\begin{alltt}
\hlstd{ricker} \hlkwb{<-} \hlkwd{ricker}\hlstd{()}
\end{alltt}
\end{kframe}
\end{knitrout}

We can plot the data:
\begin{knitrout}\small
\definecolor{shadecolor}{rgb}{0.969, 0.969, 0.969}\color{fgcolor}\begin{kframe}
\begin{alltt}
\hlkwd{plot}\hlstd{(ricker)}
\end{alltt}
\end{kframe}
\end{knitrout}

\vspace{-15mm}

\begin{knitrout}\small
\definecolor{shadecolor}{rgb}{0.969, 0.969, 0.969}\color{fgcolor}

{\centering \includegraphics[width=12cm]{tmp/figureplot-ricker-1} 

}



\end{knitrout}

\end{frame}

\begin{frame}[fragile]
\bi

\item We can simulate from the model:
\ei

\begin{knitrout}\small
\definecolor{shadecolor}{rgb}{0.969, 0.969, 0.969}\color{fgcolor}\begin{kframe}
\begin{alltt}
\hlstd{simulated_ricker} \hlkwb{<-} \hlkwd{simulate}\hlstd{(ricker)}
\end{alltt}
\end{kframe}
\end{knitrout}

\bi

\item What kind of object have we created?
\ei
\begin{knitrout}\small
\definecolor{shadecolor}{rgb}{0.969, 0.969, 0.969}\color{fgcolor}\begin{kframe}
\begin{alltt}
\hlkwd{class}\hlstd{(simulated_ricker)}
\end{alltt}
\begin{verbatim}
## [1] "pomp"
## attr(,"package")
## [1] "pomp"
\end{verbatim}
\end{kframe}
\end{knitrout}


\end{frame}    

\begin{frame}[fragile]  

\frametitle{What is a generic function?}

\bi

\item How does the concept of a \myemph{generic function} fit in with the following related concepts,

\item \myemph{object-oriented programming}

\item assigning a \myemph{class} to an object.

\item \myemph{overloading} of functions or operators.

\item \myemph{inheritance} between classes, when one class extends another.

\item How does object-oriented programming work in R? How is this similar or different from any other environment in which you have seen object-oriented programming?

\item For current purposes, we don't need to be experts in object-oriented programming in R. However, we should be familiar with some R object-orientated basics.

\ei

\end{frame}

\begin{frame}[fragile]

\bi

\item \myemph{S3 classes} (\url{http://adv-r.had.co.nz/OO-essentials.html#s3})

\item \myemph{S4 classes} (\url{http://adv-r.had.co.nz/S4.htm})

\item We should be able to recognize when code we are using employs S3 and S4 classes.

\item We should know where to turn to for help if we find ourselves needing to know more details about how these work.

\item \package{pomp} uses the S4 class system, so that system is of more immediate relevance. Many older R packages use S3 classes.

\ei

\end{frame}    

\begin{frame}[fragile] 

\begin{knitrout}\small
\definecolor{shadecolor}{rgb}{0.969, 0.969, 0.969}\color{fgcolor}\begin{kframe}
\begin{alltt}
\hlkwd{plot}\hlstd{(simulated_ricker)}
\end{alltt}
\end{kframe}
\end{knitrout}

\vspace{-10mm}

\begin{knitrout}\small
\definecolor{shadecolor}{rgb}{0.969, 0.969, 0.969}\color{fgcolor}

{\centering \includegraphics[width=12cm]{tmp/figureplot_simulated_ricker-1} 

}



\end{knitrout}

\vspace{-3mm}

\bi
\item This \package{pomp} representation uses \code{N} for our variable \code{P_n} 
\ei

\myquestion. Why do we see more time series in the simulated \code{pomp} object?

\answer{\vspace{30mm}}{todo}

\end{frame}

\begin{frame}[fragile]

\frametitle{Different formats for simulation}


\bi
\item We can turn a \code{pomp} object into a data frame:
\ei
\begin{knitrout}\small
\definecolor{shadecolor}{rgb}{0.969, 0.969, 0.969}\color{fgcolor}\begin{kframe}
\begin{alltt}
\hlstd{y} \hlkwb{<-} \hlkwd{as.data.frame}\hlstd{(ricker)}
\hlkwd{head}\hlstd{(y,}\hlnum{3}\hlstd{)}
\end{alltt}
\begin{verbatim}
##   time  y
## 1    0 68
## 2    1  2
## 3    2 87
\end{verbatim}
\begin{alltt}
\hlkwd{head}\hlstd{(}\hlkwd{simulate}\hlstd{(ricker,}\hlkwc{format}\hlstd{=}\hlstr{"data.frame"}\hlstd{))}
\end{alltt}
\begin{verbatim}
##   time .id            N         e   y
## 1    0   1 7.000000e+00 0.0000000  85
## 2    1   1 4.326721e-01 0.4163146   1
## 3    2   1 1.406521e+01 0.1141514 147
## 4    3   1 7.566248e-04 0.4348591   0
## 5    4   1 4.084348e-02 0.1893917   0
## 6    5   1 1.794869e+00 0.0237836  21
\end{verbatim}
\end{kframe}
\end{knitrout}

\bi
\item We can also run multiple simulations simultaneously:
\ei
\begin{knitrout}\small
\definecolor{shadecolor}{rgb}{0.969, 0.969, 0.969}\color{fgcolor}\begin{kframe}
\begin{alltt}
\hlstd{x} \hlkwb{<-} \hlkwd{simulate}\hlstd{(ricker,}\hlkwc{nsim}\hlstd{=}\hlnum{10}\hlstd{)}
\hlkwd{class}\hlstd{(x)}
\end{alltt}
\begin{verbatim}
## [1] "pompList"
## attr(,"package")
## [1] "pomp"
\end{verbatim}
\begin{alltt}
\hlkwd{sapply}\hlstd{(x,class)}
\end{alltt}
\begin{verbatim}
##  [1] "pomp" "pomp" "pomp" "pomp" "pomp" "pomp" "pomp" "pomp"
##  [9] "pomp" "pomp"
\end{verbatim}
\begin{alltt}
\hlstd{x} \hlkwb{<-} \hlkwd{simulate}\hlstd{(ricker,}\hlkwc{nsim}\hlstd{=}\hlnum{10}\hlstd{,}\hlkwc{format}\hlstd{=}\hlstr{"data.frame"}\hlstd{)}
\hlkwd{head}\hlstd{(x,}\hlnum{3}\hlstd{)}
\end{alltt}
\begin{verbatim}
##   time .id N e  y
## 1    0   1 7 0 63
## 2    0   2 7 0 76
## 3    0   3 7 0 70
\end{verbatim}
\end{kframe}
\end{knitrout}

\end{frame}

\begin{frame}[fragile]

\begin{knitrout}\small
\definecolor{shadecolor}{rgb}{0.969, 0.969, 0.969}\color{fgcolor}\begin{kframe}
\begin{alltt}
\hlstd{x} \hlkwb{<-} \hlkwd{simulate}\hlstd{(ricker,}\hlkwc{nsim}\hlstd{=}\hlnum{9}\hlstd{,}\hlkwc{format}\hlstd{=}\hlstr{"data.frame"}\hlstd{,}\hlkwc{include.data}\hlstd{=}\hlnum{TRUE}\hlstd{)}
\hlkwd{ggplot}\hlstd{(}\hlkwc{data}\hlstd{=x,}\hlkwd{aes}\hlstd{(}\hlkwc{x}\hlstd{=time,}\hlkwc{y}\hlstd{=y,}\hlkwc{group}\hlstd{=.id,}\hlkwc{color}\hlstd{=(.id}\hlopt{==}\hlstr{"data"}\hlstd{)))}\hlopt{+}
  \hlkwd{geom_line}\hlstd{()}\hlopt{+}\hlkwd{guides}\hlstd{(}\hlkwc{color}\hlstd{=}\hlnum{FALSE}\hlstd{)}\hlopt{+}
  \hlkwd{facet_wrap}\hlstd{(}\hlopt{~}\hlstd{.id,}\hlkwc{ncol}\hlstd{=}\hlnum{2}\hlstd{)}
\end{alltt}
\end{kframe}
\end{knitrout}

\begin{knitrout}\small
\definecolor{shadecolor}{rgb}{0.969, 0.969, 0.969}\color{fgcolor}

{\centering \includegraphics[width=12cm]{tmp/figureplot_sim_demo-1} 

}



\end{knitrout}


\end{frame}

\begin{frame}[fragile]

\frametitle{The deterministic skeleton}
\bi
\item The \myemph{deterministic skeleton} is a version of the POMP model without process noise.
\item It is generated by \code{trajectory()}.
\ei
\begin{knitrout}\small
\definecolor{shadecolor}{rgb}{0.969, 0.969, 0.969}\color{fgcolor}\begin{kframe}
\begin{alltt}
\hlstd{y} \hlkwb{<-} \hlkwd{trajectory}\hlstd{(ricker)}
\hlkwd{dim}\hlstd{(y)}
\end{alltt}
\begin{verbatim}
## [1]  2  1 51
\end{verbatim}
\begin{alltt}
\hlkwd{dimnames}\hlstd{(y)}
\end{alltt}
\begin{verbatim}
## $variable
## [1] "N" "e"
## 
## $rep
## NULL
## 
## $time
## NULL
\end{verbatim}
\end{kframe}
\end{knitrout}

\end{frame}

\begin{frame}[fragile]

\begin{knitrout}\small
\definecolor{shadecolor}{rgb}{0.969, 0.969, 0.969}\color{fgcolor}\begin{kframe}
\begin{alltt}
\hlkwd{plot}\hlstd{(}\hlkwd{time}\hlstd{(ricker),y[}\hlstr{"N"}\hlstd{,}\hlnum{1}\hlstd{,],}\hlkwc{type}\hlstd{=}\hlstr{"l"}\hlstd{)}
\end{alltt}
\end{kframe}
\end{knitrout}


\begin{knitrout}\small
\definecolor{shadecolor}{rgb}{0.969, 0.969, 0.969}\color{fgcolor}

{\centering \includegraphics[width=12cm]{tmp/figureplot_ricker_traj-1} 

}



\end{knitrout}

\end{frame}

\begin{frame}[fragile,allowframebreaks]
\frametitle{Working with model parameters}
\bi
\item \code{ricker} has parameters associated with it:
\ei
\begin{knitrout}\small
\definecolor{shadecolor}{rgb}{0.969, 0.969, 0.969}\color{fgcolor}\begin{kframe}
\begin{alltt}
\hlkwd{coef}\hlstd{(ricker)}
\end{alltt}
\begin{verbatim}
##        r    sigma      phi        c      N_0      e_0 
## 44.70118  0.30000 10.00000  1.00000  7.00000  0.00000
\end{verbatim}
\end{kframe}
\end{knitrout}
\bi
\item These are the parameters at which the simulations and deterministic trajectory computations above were done.

\item We can run these at different parameters:
\ei
\begin{knitrout}\small
\definecolor{shadecolor}{rgb}{0.969, 0.969, 0.969}\color{fgcolor}\begin{kframe}
\begin{alltt}
\hlstd{theta} \hlkwb{<-} \hlkwd{coef}\hlstd{(ricker)}
\hlstd{theta[}\hlkwd{c}\hlstd{(}\hlstr{"r"}\hlstd{,}\hlstr{"N.0"}\hlstd{)]} \hlkwb{<-} \hlkwd{c}\hlstd{(}\hlnum{5}\hlstd{,}\hlnum{3}\hlstd{)}
\hlstd{y} \hlkwb{<-} \hlkwd{trajectory}\hlstd{(ricker,}\hlkwc{params}\hlstd{=theta)}
\hlkwd{plot}\hlstd{(}\hlkwd{time}\hlstd{(ricker),y[}\hlstr{"N"}\hlstd{,}\hlnum{1}\hlstd{,],}\hlkwc{type}\hlstd{=}\hlstr{"l"}\hlstd{)}
\end{alltt}
\end{kframe}

{\centering \includegraphics[width=10cm]{tmp/figureplot_at_different_parameters-1} 

}


\begin{kframe}\begin{alltt}
\hlstd{x} \hlkwb{<-} \hlkwd{simulate}\hlstd{(ricker,}\hlkwc{params}\hlstd{=theta)}
\hlkwd{plot}\hlstd{(x,}\hlkwc{var}\hlstd{=}\hlstr{"y"}\hlstd{)}
\end{alltt}
\end{kframe}

{\centering \includegraphics[width=10cm]{tmp/figureplot_at_different_parameters-2} 

}



\end{knitrout}

\end{frame}
\begin{frame}[fragile]
\bi
\item We can also change the parameters stored inside of \code{ricker}:
\ei
\begin{knitrout}\small
\definecolor{shadecolor}{rgb}{0.969, 0.969, 0.969}\color{fgcolor}\begin{kframe}
\begin{alltt}
\hlkwd{coef}\hlstd{(ricker,}\hlkwd{c}\hlstd{(}\hlstr{"r"}\hlstd{,}\hlstr{"N.0"}\hlstd{,}\hlstr{"sigma"}\hlstd{))} \hlkwb{<-} \hlkwd{c}\hlstd{(}\hlnum{39}\hlstd{,}\hlnum{0.5}\hlstd{,}\hlnum{1}\hlstd{)}
\hlkwd{coef}\hlstd{(ricker)}
\hlkwd{plot}\hlstd{(}\hlkwd{simulate}\hlstd{(ricker),}\hlkwc{var}\hlstd{=}\hlstr{"y"}\hlstd{)}
\end{alltt}
\end{kframe}
\end{knitrout}

\vspace{-5mm}

\begin{knitrout}\small
\definecolor{shadecolor}{rgb}{0.969, 0.969, 0.969}\color{fgcolor}\begin{kframe}
\begin{verbatim}
##     r sigma   phi     c   N_0   e_0   N.0 
##  39.0   1.0  10.0   1.0   7.0   0.0   0.5
\end{verbatim}
\end{kframe}

{\centering \includegraphics[width=12cm]{tmp/figurechange_ricker_coef-1} 

}



\end{knitrout}

\end{frame}

\begin{frame}[fragile]

It is possible to work with more than one set of parameters at a time:

\begin{knitrout}\small
\definecolor{shadecolor}{rgb}{0.969, 0.969, 0.969}\color{fgcolor}\begin{kframe}
\begin{alltt}
\hlstd{p} \hlkwb{<-} \hlkwd{parmat}\hlstd{(}\hlkwd{coef}\hlstd{(ricker),}\hlnum{500}\hlstd{)}
\hlstd{p[}\hlstr{"r"}\hlstd{,]} \hlkwb{<-} \hlkwd{seq}\hlstd{(}\hlkwc{from}\hlstd{=}\hlnum{2}\hlstd{,}\hlkwc{to}\hlstd{=}\hlnum{40}\hlstd{,}\hlkwc{length}\hlstd{=}\hlnum{500}\hlstd{)}
\hlstd{y} \hlkwb{<-} \hlkwd{trajectory}\hlstd{(ricker,}\hlkwc{params}\hlstd{=p,}\hlkwc{times}\hlstd{=}\hlnum{200}\hlopt{:}\hlnum{1000}\hlstd{)}
\hlkwd{matplot}\hlstd{(p[}\hlstr{"r"}\hlstd{,],y[}\hlstr{"N"}\hlstd{,,],}\hlkwc{pch}\hlstd{=}\hlstr{"."}\hlstd{,}\hlkwc{col}\hlstd{=}\hlstr{'black'}\hlstd{,}\hlkwc{xlab}\hlstd{=}\hlstr{'r'}\hlstd{,}\hlkwc{ylab}\hlstd{=}\hlstr{'N'}\hlstd{,}\hlkwc{log}\hlstd{=}\hlstr{'x'}\hlstd{)}
\end{alltt}
\end{kframe}
\end{knitrout}

\vspace{-1mm}

\begin{knitrout}\small
\definecolor{shadecolor}{rgb}{0.969, 0.969, 0.969}\color{fgcolor}

{\centering \includegraphics[width=8cm]{tmp/figurebifdiag-1} 

}



\end{knitrout}

This is a bifurcation diagram (\url{https://en.wikipedia.org/wiki/Bifurcation_diagram}) for the Ricker equation.

\end{frame}

\begin{frame}[fragile]

\myquestion. How do you interpret this bifurcation diagram? 
\bi
\item What does it mean when the single line for small values of $r$ splits into a double line, around $r=0.8$? 

\answer{\vspace{25mm}}{todo}

\item What does it mean when solid vertical lines appear, around $r=18$?

\answer{\vspace{25mm}}{todo}

\item A bifurcation diagram like this can only be computed for a deterministic map. However, the bifurcation diagram for the deterministic skeleton can be useful to help understand a stochastic process. We'll see an example later in this chapter.

\ei

\end{frame}

\begin{frame}[fragile]

\frametitle{Burn-in}

\bi

\item Look at the R code for the bifurcation diagram. Notice that the first 200 iterations of the Ricker map are discarded, by setting \code{times=200:1000}. This is a technique called \myemph{burn-in}. 

\item This is used when aiming to simulate the steady state of a dynamic system, ignoring \myemph{transient behavior} from initial conditions.


%\item More information on manipulating and extracting information from \code{pomp} objects can be viewed in the help pages (\code{methods?pomp}).

\ei


\end{frame}    

\begin{frame}[fragile] 

\frametitle{Inference algorithms in \package{pomp}}

\bi
\item \package{pomp} provides a wide range of inference algorithms.
We'll learn about these in detail soon, but for now, let's just look at some of their general features.

\item The \code{pfilter} function runs a simple \myemph{particle filter}, which is a Monte Carlo algorithm that can be used to evaluate the likelihood at a particular set of parameters.
One uses the \code{Np} argument to specify the number of particles to use:
\ei
\begin{knitrout}\small
\definecolor{shadecolor}{rgb}{0.969, 0.969, 0.969}\color{fgcolor}\begin{kframe}
\begin{alltt}
\hlstd{pf} \hlkwb{<-} \hlkwd{pfilter}\hlstd{(ricker,}\hlkwc{Np}\hlstd{=}\hlnum{1000}\hlstd{)}
\hlkwd{class}\hlstd{(pf)}
\end{alltt}
\begin{verbatim}
## [1] "pfilterd_pomp"
## attr(,"package")
## [1] "pomp"
\end{verbatim}
\begin{alltt}
\hlkwd{plot}\hlstd{(pf)}
\end{alltt}
\end{kframe}

{\centering \includegraphics[width=\maxwidth]{tmp/figurepfilter1-1} 

}


\begin{kframe}\begin{alltt}
\hlkwd{logLik}\hlstd{(pf)}
\end{alltt}
\begin{verbatim}
## [1] -157.7213
\end{verbatim}
\end{kframe}
\end{knitrout}
\bi
\item Note that \code{pfilter} returns an object of class \code{pfilterd.pomp}.
This is the general rule: inference algorithms return objects that are \code{pomp} objects with additional information.

\item We can run the same particle filter computation again by doing
\ei
\begin{knitrout}\small
\definecolor{shadecolor}{rgb}{0.969, 0.969, 0.969}\color{fgcolor}\begin{kframe}
\begin{alltt}
\hlstd{pf} \hlkwb{<-} \hlkwd{pfilter}\hlstd{(pf)}
\hlkwd{logLik}\hlstd{(pf)}
\end{alltt}
\begin{verbatim}
## [1] -157.9107
\end{verbatim}
\end{kframe}
\end{knitrout}
\bi
\item Because the particle filter is a Monte Carlo algorithm, we get a slightly different estimate of the log likelihood.

\item By default, running \code{pfilter} on a \code{pfilterd.pomp} object causes the computation to be re-run with the same parameters as before.
Any additional arguments we add override these defaults.
This is the general rule in \package{pomp}.
For example, to repeat the particle filter with only 100 particles:
\ei
\begin{knitrout}\small
\definecolor{shadecolor}{rgb}{0.969, 0.969, 0.969}\color{fgcolor}\begin{kframe}
\begin{alltt}
\hlstd{pf} \hlkwb{<-} \hlkwd{pfilter}\hlstd{(pf,}\hlkwc{Np}\hlstd{=}\hlnum{100}\hlstd{)}
\hlkwd{logLik}\hlstd{(pf)}
\end{alltt}
\begin{verbatim}
## [1] -160.3186
\end{verbatim}
\end{kframe}
\end{knitrout}

\end{frame}    

\begin{frame}[fragile] 
\frametitle{Building a custom \code{pomp} object}

A real \package{pomp} data analysis begins with constructing one or more \code{pomp} objects to hold the data and the model or models under consideration.
We illustrate this process a dataset of the abundance of the great tit ({\it Parus major} in Wytham Wood, near Oxford \citep{mccleery91}.

First, we load and plot the data:
\begin{knitrout}\small
\definecolor{shadecolor}{rgb}{0.969, 0.969, 0.969}\color{fgcolor}\begin{kframe}
\begin{alltt}
\hlstd{dat} \hlkwb{<-} \hlkwd{read.csv}\hlstd{(}\hlstr{"parus.csv"}\hlstd{)}
\hlkwd{head}\hlstd{(dat)}
\end{alltt}
\begin{verbatim}
##   year pop
## 1 1960 148
## 2 1961 258
## 3 1962 185
## 4 1963 170
## 5 1964 267
## 6 1965 239
\end{verbatim}
\begin{alltt}
\hlkwd{plot}\hlstd{(pop}\hlopt{~}\hlstd{year,}\hlkwc{data}\hlstd{=dat,}\hlkwc{type}\hlstd{=}\hlstr{'o'}\hlstd{)}
\end{alltt}
\end{kframe}

{\centering \includegraphics[width=\maxwidth]{tmp/figureparus-data-1} 

}



\end{knitrout}

\end{frame}

\begin{frame}[fragile]
Let's suppose that we want to fit the stochastic Ricker model discussed above to these data.
\bi
\item The call to construct a \code{pomp} object is, naturally enough, \code{pomp}.
\item Documentation on this function can be had by doing \code{?pomp}. 
\item Do \code{class?pomp} to get documentation on the \code{pomp} class.
\item Learn about the various things you can do once you have a \code{pomp} object by doing \code{methods?pomp} and following the links therein.
\item Read an overview of the package as a whole with links to its main features by doing \code{package?pomp}.
\item A complete index of the functions in \package{pomp} is returned by the command \code{library(help=pomp)}.
\item The home page for the \code{pomp} project is (http://kingaa.github.io/pomp);
there you have access to the complete source code, manuals, mailing lists, etc.
\ei

\end{frame}

\begin{frame}[fragile]

The simplest \code{pomp} object has only the data:
\begin{knitrout}\small
\definecolor{shadecolor}{rgb}{0.969, 0.969, 0.969}\color{fgcolor}\begin{kframe}
\begin{alltt}
\hlstd{parus} \hlkwb{<-} \hlkwd{pomp}\hlstd{(dat,}\hlkwc{times}\hlstd{=}\hlstr{"year"}\hlstd{,}\hlkwc{t0}\hlstd{=}\hlnum{1959}\hlstd{)}
\end{alltt}
\end{kframe}
\end{knitrout}
The \code{times} argument specifies that the column labelled "year" gives the measurement times;
\code{t0} is the "zero-time", the time at which the state process will be initialized.
We've set it to one year prior to the beginning of the data.
\end{frame}

\begin{frame}[fragile]
\bi
\item  We can now plot the data:
\ei
\begin{knitrout}\small
\definecolor{shadecolor}{rgb}{0.969, 0.969, 0.969}\color{fgcolor}\begin{kframe}
\begin{alltt}
\hlkwd{plot}\hlstd{(parus)}
\end{alltt}
\end{kframe}
\end{knitrout}
\begin{knitrout}\small
\definecolor{shadecolor}{rgb}{0.969, 0.969, 0.969}\color{fgcolor}

{\centering \includegraphics[width=12cm]{tmp/figureparus-plot1-1} 

}



\end{knitrout}

\end{frame}

\begin{frame}[fragile]

\frametitle{Adding in the deterministic skeleton}

We can add the Ricker model deterministic skeleton to the \code{parus} \code{pomp} object.
Since the Ricker model is a discrete-time model, its skeleton is a map that takes $P_n$ to $P_{n+1}$ according to the Ricker model equation
\[
P_{n+1} = r\,P_{n}\,\exp(-P_{n}).
\]
We provide this to \code{pomp} in the form of a \code{Csnippet}, a little snippet of C code that performs the computation.
\begin{knitrout}\small
\definecolor{shadecolor}{rgb}{0.969, 0.969, 0.969}\color{fgcolor}\begin{kframe}
\begin{alltt}
\hlstd{skel} \hlkwb{<-} \hlkwd{Csnippet}\hlstd{(}\hlstr{"DN = r*N*exp(-N);"}\hlstd{)}
\end{alltt}
\end{kframe}
\end{knitrout}
We then add this to the \code{pomp} object:
\begin{knitrout}\small
\definecolor{shadecolor}{rgb}{0.969, 0.969, 0.969}\color{fgcolor}\begin{kframe}
\begin{alltt}
\hlstd{parus} \hlkwb{<-} \hlkwd{pomp}\hlstd{(parus,}\hlkwc{skeleton}\hlstd{=}\hlkwd{map}\hlstd{(skel),}\hlkwc{statenames}\hlstd{=}\hlstr{"N"}\hlstd{,}\hlkwc{paramnames}\hlstd{=}\hlstr{"r"}\hlstd{)}
\end{alltt}
\end{kframe}
\end{knitrout}
Note that we have to inform \package{pomp} as to which of the variables we've referred to in \code{skel} is a state variable (\code{statenames}) and which is a parameter (\code{paramnames}).

\end{frame}

\begin{frame}[fragile]

With just the skeleton defined, we are in a position to compute the trajectories of the deterministic skeleton at any point in parameter space.
For example,

\begin{knitrout}\small
\definecolor{shadecolor}{rgb}{0.969, 0.969, 0.969}\color{fgcolor}\begin{kframe}
\begin{alltt}
\hlstd{traj} \hlkwb{<-} \hlkwd{trajectory}\hlstd{(parus,}\hlkwc{params}\hlstd{=}\hlkwd{c}\hlstd{(}\hlkwc{N.0}\hlstd{=}\hlnum{1}\hlstd{,}\hlkwc{r}\hlstd{=}\hlnum{12}\hlstd{),} \hlkwc{format}\hlstd{=}\hlstr{"data.frame"}\hlstd{)}
\hlkwd{ggplot}\hlstd{(}\hlkwc{data}\hlstd{=traj,}\hlkwd{aes}\hlstd{(}\hlkwc{x}\hlstd{=year,}\hlkwc{y}\hlstd{=N))}\hlopt{+}\hlkwd{geom_line}\hlstd{()}
\end{alltt}
\end{kframe}

{\centering \includegraphics[width=8cm]{tmp/figureparus-first-traj-1} 

}



\end{knitrout}

\end{frame}

\begin{frame}[fragile]
\bi
\item
Note that the dynamics become very different if the skeleton is considered as the derivative of a differential equation rather than as a discrete time map. It is harder to get chaotic dynamics in a continuous time system. 
\ei

\begin{knitrout}\small
\definecolor{shadecolor}{rgb}{0.969, 0.969, 0.969}\color{fgcolor}\begin{kframe}
\begin{alltt}
\hlstd{parus2} \hlkwb{<-} \hlkwd{pomp}\hlstd{(parus,}\hlkwc{skeleton}\hlstd{=}\hlkwd{vectorfield}\hlstd{(skel),}\hlkwc{statenames}\hlstd{=}\hlstr{"N"}\hlstd{,}\hlkwc{paramnames}\hlstd{=}\hlstr{"r"}\hlstd{)}
\hlstd{traj2} \hlkwb{<-} \hlkwd{trajectory}\hlstd{(parus2,}\hlkwc{params}\hlstd{=}\hlkwd{c}\hlstd{(}\hlkwc{N.0}\hlstd{=}\hlnum{1}\hlstd{,}\hlkwc{r}\hlstd{=}\hlnum{12}\hlstd{),}\hlkwc{format}\hlstd{=}\hlstr{"data.frame"}\hlstd{)}
\hlkwd{ggplot}\hlstd{(}\hlkwc{data}\hlstd{=traj2,}\hlkwd{aes}\hlstd{(}\hlkwc{x}\hlstd{=year,}\hlkwc{y}\hlstd{=N))}\hlopt{+}\hlkwd{geom_line}\hlstd{()}
\end{alltt}
\end{kframe}

{\centering \includegraphics[width=8cm]{tmp/figureparus-first-traj-vector-1} 

}



\end{knitrout}

\end{frame}   

 \begin{frame}[fragile] 

\frametitle{A note on terminology}

\bi

\item If we know the state, $x(t_0)$, of the system at time $t_0$, it makes sense to speak about the entire trajectory of the system for all $t>t_0$.

\item This is true whether we are thinking of the system as deterministic or stochastic.

\item Of course, in the former case, the trajectory is uniquely determined by $x(t_0)$, while in the stochastic case, only the probability distribution of $x(t)$, $t>t_0$ is determined.

\item In \package{pomp}, to avoid confusion, we use the term ``trajectory'' exclusively to refer to \myemph{trajectories of a deterministic process}.
Thus, the \code{trajectory} command iterates or integrates the deterministic skeleton forward in time, returning the unique trajectory determined by the specified parameters.
When we want to speak about sample paths of a stochastic process, we use the term \myemph{simulation}.

\item Accordingly, the \code{simulate} command always returns individual sample paths from the POMP.
In particular, we avoid ``simulating a set of differential equations'', preferring instead to speak of ``integrating'' the equations, or ``computing trajectories''.

\ei

\end{frame}    

\begin{frame}[fragile] 

\frametitle{Adding in the process model simulator}

\bi
\item We can add the stochastic Ricker model to \code{parus} by writing a Csnippet that simulates one realization of the stochastic process, from an arbitary time $t$ to $t+1$, given arbitrary states and parameters:
\ei
\begin{knitrout}\small
\definecolor{shadecolor}{rgb}{0.969, 0.969, 0.969}\color{fgcolor}\begin{kframe}
\begin{alltt}
\hlstd{stochStep} \hlkwb{<-} \hlkwd{Csnippet}\hlstd{(}\hlstr{"
  e = rnorm(0,sigma);
  N = r*N*exp(-N+e);
"}\hlstd{)}
\hlkwd{pomp}\hlstd{(parus,}\hlkwc{rprocess}\hlstd{=}\hlkwd{discrete.time.sim}\hlstd{(}\hlkwc{step.fun}\hlstd{=stochStep,}\hlkwc{delta.t}\hlstd{=}\hlnum{1}\hlstd{),}
     \hlkwc{paramnames}\hlstd{=}\hlkwd{c}\hlstd{(}\hlstr{"r"}\hlstd{,}\hlstr{"sigma"}\hlstd{),}\hlkwc{statenames}\hlstd{=}\hlkwd{c}\hlstd{(}\hlstr{"N"}\hlstd{,}\hlstr{"e"}\hlstd{))} \hlkwb{->} \hlstd{parus}
\end{alltt}
\end{kframe}
\end{knitrout}

\bi

\item Note that in the above, we use the \code{exp} and \code{rnorm} functions from the  C language API for {\R} 
(\url{https://cran.r-project.org/doc/manuals/R-exts.html})

\item In general any C function provided by {\R} is available to you.
\package{pomp} also provides a number of C functions that are documented in the header file, \code{pomp.h}, that is installed with the package.

\item See the \code{Csnippet} documentation (\code{?Csnippet}) to read more about how to write them.

% \item Note too that we use \code{discrete.time.sim} here because the model is a stochastic map.

% \item We specify that the time step of the discrete-time process is \code{delta.t}, here, 1 year.

\ei

\end{frame}

\begin{frame}[fragile]
\bi
\item At this point, we have what we need to simulate the stochastic Ricker model.
\ei
\begin{knitrout}\small
\definecolor{shadecolor}{rgb}{0.969, 0.969, 0.969}\color{fgcolor}\begin{kframe}
\begin{alltt}
\hlstd{sim} \hlkwb{<-} \hlkwd{simulate}\hlstd{(parus,}\hlkwc{params}\hlstd{=}\hlkwd{c}\hlstd{(}\hlkwc{N.0}\hlstd{=}\hlnum{1}\hlstd{,}\hlkwc{e.0}\hlstd{=}\hlnum{0}\hlstd{,}\hlkwc{r}\hlstd{=}\hlnum{12}\hlstd{,}\hlkwc{sigma}\hlstd{=}\hlnum{0.5}\hlstd{),}
                \hlkwc{format}\hlstd{=}\hlstr{"data.frame"}\hlstd{)}
\hlkwd{plot}\hlstd{(N}\hlopt{~}\hlstd{year,}\hlkwc{data}\hlstd{=sim,}\hlkwc{type}\hlstd{=}\hlstr{'o'}\hlstd{)}
\end{alltt}
\end{kframe}
\end{knitrout}
\begin{knitrout}\small
\definecolor{shadecolor}{rgb}{0.969, 0.969, 0.969}\color{fgcolor}

{\centering \includegraphics[width=12cm]{tmp/figurericker-first-sim-1} 

}



\end{knitrout}

\end{frame}

\begin{frame}[fragile]

\frametitle{Adding in the measurement model and parameters}

\bi

\item We complete the specification of the POMP by specifying the measurement model.
To obtain the Poisson measurement model described above, we write two Csnippets. The first simulates:
\ei
\begin{knitrout}\small
\definecolor{shadecolor}{rgb}{0.969, 0.969, 0.969}\color{fgcolor}\begin{kframe}
\begin{alltt}
\hlstd{rmeas} \hlkwb{<-} \hlkwd{Csnippet}\hlstd{(}\hlstr{"pop = rpois(phi*N);"}\hlstd{)}
\end{alltt}
\end{kframe}
\end{knitrout}
\bi
\item The second computes the likelihood of observing \code{pop} birds given a true density of \code{N}:
\ei
\begin{knitrout}\small
\definecolor{shadecolor}{rgb}{0.969, 0.969, 0.969}\color{fgcolor}\begin{kframe}
\begin{alltt}
\hlstd{dmeas} \hlkwb{<-} \hlkwd{Csnippet}\hlstd{(}\hlstr{"lik = dpois(pop,phi*N,give_log);"}\hlstd{)}
\end{alltt}
\end{kframe}
\end{knitrout}
\bi
\item Note the \code{give_log} argument. When this code is evaluated, \code{give_log} will be set to 1 if the log likelihood is desired, and 0 else.

\item We add these specifications of \code{rmeasure} and \code{dmeasure} into the \code{pomp} object:
\ei

\begin{knitrout}\small
\definecolor{shadecolor}{rgb}{0.969, 0.969, 0.969}\color{fgcolor}\begin{kframe}
\begin{alltt}
\hlkwd{pomp}\hlstd{(parus,}\hlkwc{rmeasure}\hlstd{=rmeas,}\hlkwc{dmeasure}\hlstd{=dmeas,}\hlkwc{statenames}\hlstd{=}\hlkwd{c}\hlstd{(}\hlstr{"N"}\hlstd{),}
  \hlkwc{paramnames}\hlstd{=}\hlkwd{c}\hlstd{(}\hlstr{"phi"}\hlstd{))} \hlkwb{->} \hlstd{parus}
\end{alltt}
\end{kframe}
\end{knitrout}

\end{frame}

\begin{frame}[fragile]

\bi
\item To simulate, we must add some parameters to the \code{pomp} object:
\ei
\begin{knitrout}\small
\definecolor{shadecolor}{rgb}{0.969, 0.969, 0.969}\color{fgcolor}\begin{kframe}
\begin{alltt}
\hlkwd{coef}\hlstd{(parus)} \hlkwb{<-} \hlkwd{c}\hlstd{(}\hlkwc{N.0}\hlstd{=}\hlnum{1}\hlstd{,}\hlkwc{e.0}\hlstd{=}\hlnum{0}\hlstd{,}\hlkwc{r}\hlstd{=}\hlnum{20}\hlstd{,}\hlkwc{sigma}\hlstd{=}\hlnum{0.1}\hlstd{,}\hlkwc{phi}\hlstd{=}\hlnum{200}\hlstd{)}
\hlstd{sims} \hlkwb{<-} \hlkwd{simulate}\hlstd{(parus,}\hlkwc{nsim}\hlstd{=}\hlnum{3}\hlstd{,}\hlkwc{format}\hlstd{=}\hlstr{"data.frame"}\hlstd{,}
  \hlkwc{include.data}\hlstd{=}\hlnum{TRUE}\hlstd{)}
\hlkwd{ggplot}\hlstd{(}\hlkwc{data}\hlstd{=sims,}\hlkwc{mapping}\hlstd{=}\hlkwd{aes}\hlstd{(}\hlkwc{x}\hlstd{=year,}\hlkwc{y}\hlstd{=pop))}\hlopt{+}\hlkwd{geom_line}\hlstd{()}\hlopt{+}
  \hlkwd{facet_wrap}\hlstd{(}\hlopt{~}\hlstd{.id)}
\end{alltt}
\end{kframe}

{\centering \includegraphics[width=10cm]{tmp/figurericker-add-params-1} 

}



\end{knitrout}

\end{frame}    

\begin{frame}[fragile] 

\frametitle{Exercise: Ricker model parameters}
\bi
\item Fiddle with the parameters to try and make the simulations look more like the data.

\item This will help you build some intuition for what the various parameters do.
\ei

\end{frame}

\begin{frame}

\frametitle{Exercise: reformulating the Ricker model}
\bi
\item Reparameterize the Ricker model so that the scaling of $P$ is explicit:
$$P_{n+1} = r\,P_{n}\,\exp\left(-\frac{P_{n}}{K}\right).$$

\item Modify the \code{pomp} object we created above to reflect this reparameterization.

\item Modify the measurement model so that
$$\mathrm{pop}_n \sim \mathrm{Negbin}(\phi\,P_n,k),$$
i.e., $\mathrm{pop}_n$ is negative-binomially distributed with mean $\phi\,P_t$ and clumping parameter $k$.

\item See \code{?NegBinomial} for documentation on the negative binomial distribution and the {\R} Extensions Manual section on distribution functions (\url{https://cran.r-project.org/doc/manuals/R-exts.html}) for information on how to access these in C.

\ei

\end{frame}

\begin{frame}[fragile]

\frametitle{Exercise: The Beverton-Holt model}

\bi

\item Construct a \code{pomp} object for the {\it Parus major} data and the \myemph{stochastic Beverton-Holt} model,
$$P_{n+1} = \frac{a\,P_n}{1+b\,P_n}\,\varepsilon_n,$$
where $a$ and $b$ are parameters and
$$\varepsilon_t \sim \mathrm{Lognormal}(-\tfrac{1}{2}\sigma^2,\sigma^2).$$

\item Assume the same measurement model as we used for the Ricker model.

\ei

\end{frame} 


\begin{frame}[fragile]
\frametitle{Acknowledgments and License}

\bi
\item These notes build on previous versions at \url{ionides.github.io/531w16} and \url{ionides.github.io/531w18}. 
\item Those notes draw on material developed for a short course on Simulation-based Inference for Epidemiological Dynamics (\url{http://kingaa.github.io/sbied/}) by Aaron King and Edward Ionides, taught at the University of Washington Summer Institute in Statistics and Modeling in Infectious Diseases, from 2015 through 2018.
\item
Licensed under the Creative Commons attribution-noncommercial license, \url{http://creativecommons.org/licenses/by-nc/3.0/}.
Please share and remix noncommercially, mentioning its origin.  
\includegraphics[width=2cm]{cc-by-nc.png}
\ei

\end{frame}


%\begin{frame}[allowframebreaks]
%\frametitle{References}
%\bibliography{notes03.bib}
%\end{frame}

\end{document}

